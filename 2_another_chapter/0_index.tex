%!TEX root = ../document.tex

% this file is called up by thesis.tex
% content in this file will be fed into the main document

%: ----------------------- Introduction file header -----------------------


\chapter{Otro Capítulo}
\label{cha:otro_capitulo}

% the code below specifies where the figures are stored
\ifpdf
    \graphicspath{{2_another_chapter/figures/PNG/}{2_another_chapter/figures/PDF/}{2_another_chapter/figures/}}
\else
    \graphicspath{{2_another_chapter/figures/EPS/}{2_another_chapter/figures/}}
\fi


%------------------------------------------------------------------------- 

Tras lo explicado en el capítulo \ref{cha:introduccion}, y más concretamente en la sección \ref{sec:motivacion}... Lorem ipsum dolor sit amet, consectetur adipiscing elit. Aenean placerat, tortor
quis adipiscing malesuada, velit tellus viverra ligula, non fermentum nunc erat
fermentum metus. Nulla et dapibus erat. Proin accumsan tristique odio, vel
fermentum leo porttitor in. Aliquam eu urna turpis. Suspendisse ac eleifend
enim. Vivamus a magna at felis consequat auctor vitae vel erat. Donec faucibus
justo vel odio pulvinar id mattis arcu vulputate. Cras risus nisi, imperdiet nec
hendrerit quis, mattis eu velit. Nunc sit amet ligula metus. Morbi magna enim,
adipiscing mollis interdum quis, bibendum id lectus. 

Lorem ipsum dolor sit amet, consectetur adipiscing elit. Aenean placerat, tortor
quis adipiscing malesuada, velit tellus viverra ligula, non fermentum nunc erat
fermentum metus. Nulla et dapibus erat. Proin accumsan tristique odio, vel
fermentum leo porttitor in. Aliquam eu urna turpis. Suspendisse ac eleifend
enim. Vivamus a magna at felis consequat auctor vitae vel erat. Donec faucibus
justo vel odio pulvinar id mattis arcu vulputate. Cras risus nisi, imperdiet nec
hendrerit quis, mattis eu velit. Nunc sit amet ligula metus. Morbi magna enim,
adipiscing mollis interdum quis, bibendum id lectus. 

Como se muestra en la \tabl{tab:example} \ldots

\begin{table}
\center
\caption{Una tabla de ejemplo}
\begin{tabular}{|r|l|}
  \hline
  7C0 & hexadecimal \\
  3700 & octal \\ \cline{2-2}
  11111000000 & binary \\
  \hline \hline
  1984 & decimal \\
  \hline
\end{tabular}
\label{tab:example}
\end{table}

Sed adipiscing justo fermentum tortor laoreet sed vestibulum tortor mattis.
Vivamus a enim augue. Donec nunc metus, facilisis vitae accumsan et,
pellentesque non lorem. Curabitur porta iaculis diam vitae varius. Duis at diam
in urna fringilla egestas a ac felis. Aliquam semper malesuada orci, id vehicula
mauris varius. Facilisis vitae accumsan et, pellentesque non lorem. Curabitur 
porta iaculis diam vitae varius. Duis at diam in urna fringilla egestas a ac 
felis. Aliquam semper malesuada orci, id vehicula mauris varius.


% ----------------------------------------------------------------------
% ----------------------------------------------------------------------
%              Latex PhD template for the University of Deusto
% ----------------------------------------------------------------------


%: Style file for Latex
% Most style definitions are in the external file PhDthesisPSnPDF.
% In this template package, it can be found in ./Latex/Classes/
%\documentclass[twoside,12pt]{Latex/Classes/PhDthesisPSnPDF}
\documentclass[oneside,12pt]{Latex/Classes/PhDthesisPSnPDF}

%\usepackage[lined, boxed]{algorithm2e} % for algorithm figures creation.

%: Macro file for Latex
% Macros help you summarise frequently repeated Latex commands.
% Here, they are placed in an external file /Latex/Macros/MacroFile1.tex
% An macro that you may use frequently is the figuremacro (see introduction.tex)
\include{Latex/Macros/Macros}


%: ----------------------------------------------------------------------
%:                  TITLE PAGE: name, degree,..
% ----------------------------------------------------------------------
% below is to generate the title page with crest and author name

% Title of the dissertation
\newcommand{\thesistitle}{Título del trabajo}

%Author name
\newcommand{\authorname}{Jon Lázaro Aduna}

%First advisor name
\newcommand{\advisorname}{Dr. Unai Aguilera Irazabal}

% if output to PDF then put the following in PDF header
\ifpdf
    \hypersetup{pdfauthor={\authorname},%
            pdftitle={\thesistitle},%
            pdfsubject={PhD Thesis},%
            pdfkeywords={keyword1, keyword2, keyword3},%
            pdfproducer={LaTeX},%
            pdfcreator={pdfLaTeX}
}
\fi

% ----------------------------------------------------------------------
% This section below defines front covert (external and internal)
% Shield logo
\crest{\includegraphics[width=2cm]{Deusto_Shield}}
% Full logo
%\crest{\includegraphics[width=6cm]{UDeusto}}
\university{Universidad de Deusto}
\degree{}
\author{\authorname} 
\collegeordept{lazaro.jon@gmail.com}
\textadvisor{Supervisor: }
\advisor{\advisorname}
\advisortwo{}
\textsignaturecandidate{El doctorando}
\textsignatureadvisor{El director}
\cityofbirth{Bilbao}
\degreedate{octubre de 2013}

% ----------------------------------------------------------------------
       
% turn of those nasty overfull and underfull hboxes
\hbadness=10000
\hfuzz=50pt

%trying to avoid orphan and widow lines
\widowpenalty10000
\clubpenalty10000

% hyphenation
%\hyphenation{so-me-thing}

%: --------------------------------------------------------------
%:                  FRONT MATTER: dedications, abstract,..
% --------------------------------------------------------------

\begin{document}

\selectlanguage{spanish}
\MejorarTraducciones

% sets line spacing
%\onehalfspacing

% Watermark
%\watermark{DRAFT	DRAFT	DRAFT	DRAFT	DRAFT	DRAFT	DRAFT	DRAFT	DRAFT}


%: ----------------------- generate cover page ------------------------

\maketitle  % command to print the title page with above variables

% Title back
%\include{0_frontmatter/title_back}


%: ----------------------- abstract ------------------------

% Your institution may have specific regulations if you need an abstract and
% where it is to be placed in the document. The default here is just after
% title.

% The original template provides and abstractseparate environment, if your
% institution requires them to be separate. I think it's easier to print the
% abstract from the complete thesis by restricting printing to the relevant
% page.
% \begin{abstractseparate}
%   \input{Abstract/abstract}
% \end{abstractseparate}


%: ----------------------- tie in front matter ------------------------

% The frontmatter text starts here
\frontmatter

%\include{0_frontmatter/dedication}
%\include{0_frontmatter/abstract}
%\include{0_frontmatter/acknowledgement}

% As abstract contains various languages we set the main language again
\selectlanguage{spanish}
\MejorarTraducciones

%: ----------------------- contents ------------------------

\setcounter{secnumdepth}{5} % organisational level that receives a numbers
\setcounter{tocdepth}{5}    % print table of contents for level 3

%You can also add extra lines to the ToC or to force extra unnumbered section
%headings to be included. For example, if you wanted to add an entry called
%Preface, and you didn't want the Preface to be numbered, you'd use these
%commands:
%\ subsection*{Preface}
%\addcontentsline{toc}{subsection}{Preface} 

%\cleardoublepage
\pdfbookmark[0]{\contentsname}{toc}
\tableofcontents            % print the table of contents
							% levels are:
							% 0 - chapter
							% 1 - section
							% 2 - subsection
							% 3 - subsection
%: ----------------------- list of figures/tables ------------------------

\listoffigures	% print list of figures
\listoftables  % print list of tables

%: ----------------------- glossary ------------------------

% Tie in external source file for definitions: /0_frontmatter/glossary.tex
% Glossary entries can also be defined in the main text. See glossary.tex
%\include{0_frontmatter/glossary}

%\begin{multicols}{2} % \begin{multicols}{#columns}[header text][space]
%\begin{footnotesize} % scriptsize(7) < footnotesize(8) < small (9) < normal (10)

%\printnomenclature[1.5cm] % [] = distance between entry and description
%\label{sec:glossary} % target name for links to glossary

%\end{footnotesize}
%\end{multicols}

%defined commands
%\newcommand{\edistancia}[0]{\emph{distancia estimada}}

%: --------------------------------------------------------------
%:                  MAIN DOCUMENT SECTION
% --------------------------------------------------------------

% the main text starts here with the introduction, 1st chapter,...
\mainmatter

%\renewcommand{\chaptername}{} % uncomment to print only "1" not "Chapter 1"

%: ----------------------- subdocuments ------------------------

% Parts of the thesis are included below. Rename the files as required. But take
% care that the paths match. You can also change the order of appearance by
% moving the include commands.

%!TEX root = ../document.tex

% this file is called up by document.tex
% content in this file will be fed into the main document

%: ----------------------- Introduction file header -----------------------


\begin{savequote}[50mm]
Un viaje de mil millas comienza
con el primer paso.
\qauthor{\textsc{Lao-tsé} (570 a.C.- 490 a.C.)}
\end{savequote}

\chapter{Introducción}
\label{cha:introduccion}
% the code below specifies where the figures are stored
\ifpdf
    \graphicspath{{1_introduction/figures/PNG/}{1_introduction/figures/PDF/}{1_introduction/figures/}}
\else
    \graphicspath{{1_introduction/figures/EPS/}{1_introduction/figures/}}
\fi

%------------------------------------------------------------------------- 

Como puede deducirse del trabajo de \cite{Bailey} y, posteriormente del de
\cite{Ta}... 

% ----------------------------------------------------------------------

%!TEX root = ../document.tex

\section{Motivación}
\label{sec:motivacion}

Curabitur placerat auctor mattis. Aliquam at augue sem. Ut adipiscing aliquet
eleifend. Nullam vitae eros vitae felis consequat placerat. Ut viverra blandit
velit at facilisis. Ut mattis pretium vestibulum. Maecenas elementum, nibh
porttitor sodales pulvinar, diam arcu molestie nunc, nec auctor magna tortor
eget eros. Nam consectetur nisl at dolor volutpat feugiat. 

Como se observa en la \fig{fig:atom} \ldots

\InsertFig{atom}{fig:atom}{Esto es un átomo}{}{0.25}{}

Mauris fringilla convallis facilisis. Quisque condimentum, enim in gravida
convallis, justo felis suscipit nisi, ac sagittis turpis nibh in sem. Curabitur
lacinia ante non lacus iaculis in lacinia turpis consectetur. Nunc sollicitudin
metus sit amet libero accumsan at rhoncus turpis volutpat. Aenean nec velit at
arcu porttitor accumsan nec et neque. In quis odio nunc. Pellentesque habitant
morbi tristique senectus et netus et malesuada fames ac turpis egestas.	
%!TEX root = ../document.tex

% this file is called up by thesis.tex
% content in this file will be fed into the main document

%: ----------------------- Introduction file header -----------------------


\chapter{Otro Capítulo}
\label{cha:otro_capitulo}

% the code below specifies where the figures are stored
\ifpdf
    \graphicspath{{2_another_chapter/figures/PNG/}{2_another_chapter/figures/PDF/}{2_another_chapter/figures/}}
\else
    \graphicspath{{2_another_chapter/figures/EPS/}{2_another_chapter/figures/}}
\fi


%------------------------------------------------------------------------- 

Tras lo explicado en el capítulo \ref{cha:introduccion}, y más concretamente en la sección \ref{sec:motivacion}... Lorem ipsum dolor sit amet, consectetur adipiscing elit. Aenean placerat, tortor
quis adipiscing malesuada, velit tellus viverra ligula, non fermentum nunc erat
fermentum metus. Nulla et dapibus erat. Proin accumsan tristique odio, vel
fermentum leo porttitor in. Aliquam eu urna turpis. Suspendisse ac eleifend
enim. Vivamus a magna at felis consequat auctor vitae vel erat. Donec faucibus
justo vel odio pulvinar id mattis arcu vulputate. Cras risus nisi, imperdiet nec
hendrerit quis, mattis eu velit. Nunc sit amet ligula metus. Morbi magna enim,
adipiscing mollis interdum quis, bibendum id lectus. 

Lorem ipsum dolor sit amet, consectetur adipiscing elit. Aenean placerat, tortor
quis adipiscing malesuada, velit tellus viverra ligula, non fermentum nunc erat
fermentum metus. Nulla et dapibus erat. Proin accumsan tristique odio, vel
fermentum leo porttitor in. Aliquam eu urna turpis. Suspendisse ac eleifend
enim. Vivamus a magna at felis consequat auctor vitae vel erat. Donec faucibus
justo vel odio pulvinar id mattis arcu vulputate. Cras risus nisi, imperdiet nec
hendrerit quis, mattis eu velit. Nunc sit amet ligula metus. Morbi magna enim,
adipiscing mollis interdum quis, bibendum id lectus. 

Como se muestra en la \tabl{tab:example} \ldots

\begin{table}
\center
\caption{Una tabla de ejemplo}
\begin{tabular}{|r|l|}
  \hline
  7C0 & hexadecimal \\
  3700 & octal \\ \cline{2-2}
  11111000000 & binary \\
  \hline \hline
  1984 & decimal \\
  \hline
\end{tabular}
\label{tab:example}
\end{table}

Sed adipiscing justo fermentum tortor laoreet sed vestibulum tortor mattis.
Vivamus a enim augue. Donec nunc metus, facilisis vitae accumsan et,
pellentesque non lorem. Curabitur porta iaculis diam vitae varius. Duis at diam
in urna fringilla egestas a ac felis. Aliquam semper malesuada orci, id vehicula
mauris varius. Facilisis vitae accumsan et, pellentesque non lorem. Curabitur 
porta iaculis diam vitae varius. Duis at diam in urna fringilla egestas a ac 
felis. Aliquam semper malesuada orci, id vehicula mauris varius.


% ----------------------------------------------------------------------	

% --------------------------------------------------------------
%:                  BACK MATTER: appendices, refs,..
% --------------------------------------------------------------

% the back matter: appendix and references close the thesis
\backmatter

%: ----------------------- appendix ------------------------

\appendix
%!TEX root = ../document.tex

\chapter{Apéndice A}
\label{chap:appendix_publications}

% the code below specifies where the figures are stored
\ifpdf
    \graphicspath{{appendix_publications/figures/PNG/}{appendix_publications/figures/PDF/}{appendix_publications/figures/}}
\else
    \graphicspath{{appendix_publications/figures/EPS/}{appendix_publications/figures/}}
\fi


%-------------------------------------------------------------------------

Suspendisse potenti. In hac habitasse platea dictumst. Nullam mattis metus eu
quam dictum blandit. Nam nisl nibh, mattis eget laoreet at, tincidunt quis
turpis. Nulla ac augue in lectus suscipit ultrices id ac odio. Ut a purus eget
nulla adipiscing dapibus a vitae mi. Vestibulum tempor odio a ipsum sollicitudin
porttitor. Phasellus porta metus a sem pellentesque eget vulputate urna
interdum. Nam euismod odio at libero tempus id eleifend ipsum viverra. In hac
habitasse platea dictumst.

% ------------------------------------------------------------------------


%: ----------------------- bibliography ------------------------

% The section below defines how references are listed and formatted The default
% below is 2 columns, small font, complete author names. Entries are also linked
% back to the page number in the text and to external URL if provided in the
% BibTex file.


% Original version:

% PhDbiblio-url2 = names small caps, title bold & hyperlinked, link to page 
%\begin{multicols}{2} % \begin{multicols}{ # columns}[ header text][ space]
%\begin{tiny} % tiny(5) < scriptsize(7) < footnotesize(8) < small (9)
%
%\bibliographystyle{Latex/Classes/PhDbiblio-url2} % Title is link if provided
%\renewcommand{\bibname}{References} % changes the header; default: Bibliography
%
%\bibliography{9_backmatter/references} % adjust this to fit your BibTex file
%
%\end{tiny}
%\end{multicols}

% Show all bibliography entries
%\nocite*


% If we want bibliography backreference, use unsrt first and the desidered one after

%\bibliographystyle{unsrt} % Defines the bibliography style

%\bibliographystyle{alpha} % Defines the bibliography style
\bibliographystyle{Latex/StyleBST/TeXiS}

%\bibliographystyle{apa-good} % Defines the bibliography style
%\bibliographystyle{natbib} % Defines the bibliography style

%\bibliographystyle{plainurl}

%\renewcommand{\bibname}{References} % changes the header; default: Bibliography

%To include the references/works cited/bibliography in your Table of Contents,
%right before the bibliography command, use the command
%\addcontentsline{toc}{section}{References}

\bibliography{9_backmatter/references} % adjust this to fit your BibTex file

% --------------------------------------------------------------
% Various bibliography styles exit. Replace above style as desired.

% in-text refs: (1) (1; 2)
% ref list: alphabetical; author(s) in small caps; initials last name; page(s)
%\bibliographystyle{Latex/Classes/PhDbiblio-case} % title forced lower case
%\bibliographystyle{Latex/Classes/PhDbiblio-bold} % title as in bibtex but bold
%\bibliographystyle{Latex/Classes/PhDbiblio-url} % bold + www link if provided

%\bibliographystyle{Latex/Classes/jmb} % calls style file jmb.bst
% in-text refs: author (year) without brackets
% ref list: alphabetical; author(s) in normal font; last name, initials; page(s)

%\bibliographystyle{plainnat} % calls style file plainnat.bst
% in-text refs: author (year) without brackets
% (this works with package natbib)

% --------------------------------------------------------------


%: back pages
%\include{9_backmatter/back}

\end{document}
